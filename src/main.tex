%! Author = dejan
%! Date = 17/06/2022

% Preamble
\documentclass[12pt]{book}

% Packages
\usepackage[chorded]{songs}
\usepackage[utf8]{inputenc}
\usepackage[sfdefault]{roboto}
\usepackage[T1]{fontenc}
\usepackage[pdftex]{graphicx}
\usepackage[
        a5paper,
        landscape,
        left=1cm,
        right=1cm,
        top=1cm,
        bottom=1cm,
        footskip=5mm,
        headheight=0
    ]{geometry}
\usepackage{color}

% \usepackage{layout}



% Številka pesmi
\setlength{\songnumwidth}{1cm}
\renewcommand{\thesongnum}{\arabic{songnum}}
\renewcommand{\printsongnum}[1]{\LARGE#1}

% Naslov pesmi
\renewcommand{\stitlefont}{
    \rmfont\bf\Large\baselineskip=20pt\lineskiplimit=0pt
}

\setlength{\fboxsep}{3pt}


% Besedilo pesmi
\setlength{\cbarwidth}{0pt}
\renewcommand{\lyricfont}{\sffamily\small}
\renewcommand{\notebgcolor}{white}
\noversenumbers
\songcolumns{2}
\columnsep=8mm
\versesep=12pt plus 2pt minus 2pt

\renewcommand{\clineparams}{ % višina akordov
    \baselineskip=10pt
    \lineskiplimit=1pt
    \lineskip=1pt
}

% Akordi
\renewcommand{\chorusfont}{\it}
\renewcommand{\printchord}[1]{\rmfamily\bf#1}
\nosongnumbers
\MultiwordChords

% konc in začetek pesmi
\setlength{\sbarheight}{0pt}
\renewcommand\makeprelude{%
    \resettitles
    {\bfseries\thesongnum. \songtitle\par
    \nexttitle\foreachtitle{(\songtitle)\par}}%

    \colorbox{green}{
        \footnotesize\extendprelude
    }

}

\renewcommand\makepostlude{%
nodline
}


% Document
\begin{document}



    % \layout

    \begin{songs}{}


        \beginsong{Tam ob ognju našem}[by={Taborniška}]



        \beginchorus
        T\[A]am ob ognju našem si sež\[E]emo v \[E7]roke,
        Plamen neugašen nam je s\[A]rce.
        Vedno te bom lju\[A7]bil, dih gozda, \[D]šum voda,
        Tu je m\[A]oj dom in vedno b\[E]om, tukaj ra\[E7]d osta\[A]l doma.
        \endchorus

        \beginverse
        Kakor lepe sanje spomin bo na te dni,
        Ko se spomnim nanje, srce vzdrhti.
        Saj mladost je naša, kot lepa majska noc,
        Vsak dan bo lep spomin krasan, vedno lep in vedno vroč.
        \endverse

        \endsong


        \beginsong{Taborniška himna}[by={Taborniška}]

        \beginverse
        Dviga p\[C]lamen se iz \[G]ognja,
        taboriš\[G7]ca naše	\[C]ga,
        ki pod \[F]goro mirno spa\[C]va,
        sredi gozda \[G7]temneg\[C]a.
        \endverse

        \beginverse
        Tam šotori se blestijo,
        prapor sredi njih vihra
        in oznanja vsej prirodi,
        kje je tabornik doma.
        \endverse

        \beginverse
        Poslušajte bratje sestre
        gozda jelovega spev,
        pesem velike prirode,
        tihi gorski njen odmev.

        \endverse
        \endsong

    \end{songs}




\end{document}